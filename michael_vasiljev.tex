\documentclass[11pt,a4paper, serif]{moderncv}

\moderncvcolor{green}
\moderncvstyle{casual}
%\nopagenumbers{}

\usepackage{polyglossia}
\setdefaultlanguage[indentfirst=true,forceheadingpunctuation=false]{russian}
\setotherlanguages{english}

\setmainfont{PT Serif}
\setsansfont{PT Sans}
\setmonofont{PT Mono}

\name{Михаил}{Васильев}
\title{Deep learning engineer}
\born{25 апреля 1987}
\phone[mobile]{+7~(916)~198~81~83}
\email{gnu.xinm@gmail.com}
\homepage{onixlas.github.io}

\social[linkedin]{michael-vasiliev-ds} 
\social[github]{onixlas}
\social[telegram]{LaHundo}

\photo[64pt][0.4pt]{mvasiljev.jpg}

\begin{document}
\makecvtitle

\section{Опыт}
\cventry{2025—н.в.}{Старший инженер по машинному обучению}{\href{https://www.raiffeisen.ru/}{Райффайзен Банк}}{}{}{}
\cventry{2023—2025}{Старший специалист по машинному обучению}{\href{https://makves.ru/}{\textenglish{Makves}}}{}{}{
Проект: \textbf{разработка и внедрение RAG-системы}\newline{}
Инструменты: \textenglish{LangChain, Ollama, Saiga, GigaChat, HuggingFace, FastAPI, Ragas}\newline{}
\begin{itemize}
\item разработал и внедрил RAG-систему для автоматизации обработки запросов заказчиков
\item оптимизировал гиперпараметры системы с использованием библиотеки \textenglish{Ragas} и \textenglish{LLM GigaChat}\newline{}
\end{itemize}
Проект: \textbf{создание комплексного решения для обеспечения безопасности в корпоративной сети на основе неструктурированных данных}\newline{}
Инструменты: \textenglish{transformers, YOLO, PyOD, pandas, sklearn, pytorch, lightning, numpy, huggingface, onnx, fastapi, uvicorn, pyinstaller, optimum, catboost, cvat, natasha}\newline{}
\begin{itemize}
\item реализовал нейросетевой модуль для поиска нарушений закона о персональных данных, количество детектируемых классов увеличено с 14 до 36, \textenglish{accuracy} top 1 увеличена до 98.9
\item подготовил модуль для анализа содержимого отсканированных документов: поиск текста, таблиц, печатей, подписей и корпоративных бланков, количество классов увеличено с 5 до 19, mAP@.5 улучшен с .89 до .94
\item реализовал поиск чувствительных данных в текстовых файлах, добавил модуль \textenglish{NER}
\item создал ансамбль алгоритмов для поиска аномалий на табличных данных, в том числе на временных рядах
\item реализовал поиск чувствительных данных в аудио-файлах
\item собрал и организовал разметку 8 датасетов для задач классификации и \textenglish{object detection}
\end{itemize}
}

\newpage

\section{Пет-проекты}
\cventry{2024}{Тим-лид и технический эксперт}{\href{https://onixlas.github.io/projects/check_doc_ai/}{CheckDocAI}}{}{}{Проект: Телеграм-бот с ИИ модулем для контроля качества оформления документов для \href{https://gulfstream.ru/}{ООО «Гольфстрим»}, позволяет значительно сократить время на проверку и улучшить точность.\newline{}\newline{}
Инструменты: \textenglish{aiogram, YOLO, ONNX, Albumentations, CVAT}\newline{}
\begin{itemize}
\item руководил командой из двух дата-сайентистов и бекенд-разработчика, отвечал за разработку и внедрение проекта.
\item проект успешно внедрён в коммерческую эксплуатацию, ежемесячная экономия~— 40~человеко-часов.
\end{itemize}}

\section{Хакатоны}
\cventry{2024}{\textenglish{VK HSE Data Hack}}{1 место}{}{}{Хакатон по классификации новостных статей на 21 класс. В нашем решении комбинируются результаты работы небольшого классификатора на базе трансформерной архитектуры и предсказания LLM\newline{}\newline{}
Инструменты: \textenglish{transformers, Saiga3 8b, taiga dataset, streamlit}\newline{}
\begin{itemize}
\item обогатил датасет
\item подобрал \textenglish{zero-shot classification} модель
\item обучил модель-классификатор
\item обеспечил координацию работы команды
\item презентовал результаты
\end{itemize}}

\section{Доклады}
\cventry{24.05.2025}{\href{https://onixlas.github.io/talks/anomaly_phdays/}{Поиск аномалий с использованием Python: от теории к практике}}{\textenglish{Positive Hack Days}}{}{}{}
\cventry{2025}{\href{https://onixlas.github.io/talks/\#аномалии}{Серия докладов: Поиск аномалий в данных, алгоритмы}}{\textenglish{Moscow Python Meetup}}{}{}{}
\cventry{2024—2025}{\href{https://onixlas.github.io/talks/\#makves}{NLP и CV нейросети в защите данных: опыт \textenglish{Makves DCAP}}}{\textenglish{Moscow Python Meetup}}{}{}{}

\newpage

\section{Образование}
\cventry{2024}{Анализ данных на языке \textenglish{SQL}}{УЦ «Специалист»}{}{повышение квалификации}{}
\cventry{2022—2023}{\textenglish{Computer Vision Engineer}}{\textenglish{Deep Learning School} ФПМИ МФТИ}{}{профессиональная переподготовка}{}
\cventry{2022}{Специалист по \textenglish{Data Science}}{Яндекс Практикум}{}{профессиональная переподготовка}{}
\cventry{2021—2022}{Введение в искусственный интеллект и нейросети для авиационных приложений}{МАИ}{}{повышение квалификации}{}
\cventry{2005—2008}{Перевод и переводоведение}{МАИ}{}{специалитет}{}
\cventry{2003—2009}{Авиационная и ракетно-космическая теплотехника}{МАИ}{}{специалитет}{}

\section{Языки}
\cvskillentry{Русский}{5}{родной}{}{}
\cvskillentry{Английский}{4}{В2}{}{}
\cvskillentry{Немецкий}{4}{В2}{}{}
\cvskillentry{Эсперанто}{4}{В2}{}{}

\section{Навыки и технологии}
\begin{english}
	\begin{cvcolumns}
	  \cvcolumn{}{
	  	\begin{itemize}
	  		\item Deep Learning
	  		\item LLM, RAG
	  		\item NLP, NER
	  		\item Computer Vision
	  		\item Speech Recognition
	  		\item Machine learning
	  		\item Anomaly Detection
	  		\item Data analysis
	  		\item Data visualisation
	  		\item Statistics
	  	\end{itemize}}
	  \cvcolumn{}{
	  	\begin{itemize}
	  		\item Python
	  		\item SQL
	  		\item Linux
	  		\item Docker
	  		\item YOLO
	  		\item Natasha
	  		\item ONNX
	  		\item HuggingFace
	  		\item Ragas
	  		\item Ollama
	  		\item U-Net
	  		\item AirFlow
	  		\item MLFlow
	  		\item CVAT
	  		\item Plotly
	  	\end{itemize}}
	  \cvcolumn{}{\begin{itemize}
	  		\item Pytorch
	  		\item Lightning
	  		\item Pandas
	  		\item NumPy
	  		\item Sklearn
	  		\item FastAPI
	  		\item uvicorn
	  		\item PyOD
	  		\item PySAD
	  		\item Optimum
	  		\item pywin32
	  		\item CatBoost
	  		\item XGBoost
	  		\item PostgreSQL
	  		\item MySQL
	  	\end{itemize}}
	\end{cvcolumns}
\end{english}
\end{document}


%% end of file `template.tex'.
