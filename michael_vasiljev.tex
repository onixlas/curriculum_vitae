\documentclass[11pt,a4paper, serif]{moderncv}

\moderncvstyle{casual}
\moderncvcolor{green}
%\nopagenumbers{}

\usepackage{polyglossia}
\setdefaultlanguage[indentfirst=true,forceheadingpunctuation=false]{russian}
\setotherlanguages{english}

%\usepackage[scale=0.9]{geometry}
%\setlength{\footskip}{136.00005pt}                 % depending on the amount of information in the footer, you need to 

\setmainfont{PT Serif}
\setsansfont{PT Sans}
\setmonofont{PT Mono}

\name{Михаил}{Васильев}
\title{Deep learning engineer}
\born{25 апреля 1987}
\phone[mobile]{+7~(916)~198~81~83}
\email{gnu.xinm@gmail.com}
\homepage{onixlas.github.io}

\social[linkedin]{michael-vasiliev-ds} 
\social[github]{onixlas}
\social[telegram]{LaHundo}

\photo[64pt][0.4pt]{mvasiljev.jpg}

\begin{document}
\makecvtitle

\section{Опыт}
\cventry{2023—н.в.}{Старший специалист по машинному обучению}{Маквес Групп}{Москва}{}{Проект: создание комплексного решения для обеспечения безопасности в корпоративной сети на основе неструктурированных данных.\newline{}\newline{}
Инструменты: python, transformers, EfficientNet, MobileNet, YOLO, OpenCV, PIL, PyOD, pandas, sklearn, pytorch, lightning, numpy, matplotlib, plotly, huggingface, onnx, fastapi, uvicorn, pyinstaller, pywin32, optimum, airflow, mlflow, cvat, natasha, deeppavlov, BERT, whisper, Ollama.\newline{}
\begin{itemize}
\item реализовал нейросетевой модуль для поиска нарушений закона о персональных данных, количество детектируемых классов увеличено с 14 до 36, accuracy top 1 увеличена до 98.9
\item подготовил модуль для анализа содержимого отсканированных документов: поиск текста, таблиц, печатей, подписей и корпоративных бланков, количество классов увеличено с 5 до 19, mAP@.5 улучшен с .89 до .94
\item реализовал поиск чувствительных данных в текстовых файлах, добавил модуль NER
\item создал ансамбль алгоритмов для поиска аномалий на табличных данных, в том числе на временных рядах
\item реализовал поиск чувствительных данных в аудио-файлах
\item собрал и организовал разметку 8 датасетов для задач классификации и object detection
\end{itemize}}

\newpage

\section{Пет-проекты}
\cventry{2024}{Тим-лид и технический эксперт}{CheckDocAI}{Москва}{}{Проект: Телеграм-бот с ИИ модулем для контроля качества оформления документов для ООО «Гольфстрим»\newline{}\newline{}
Инструменты: aiogram, yolo, onnx, albumentations, cvat\newline{}
\begin{itemize}
\item руководил командой из двух дата-сайнтистов и бекенд-разработчика
\item проект завершён и внедрён в коммерческую эксплуатацию
\item ежемесячная экономия — 40 человеко-часов
\end{itemize}}

\section{Хакатоны}
\cventry{2024}{VK HSE Data Hack}{1 место}{Москва}{}{Хакатон по классификации новостных статей на 21 класс. В нашем решении комбинируются результаты работы небольшого классификатора на базе трансформерной архитектуры и предсказания LLM\newline{}\newline{}
Инструменты: transformers, Saiga3 8b, taiga dataset, streamlit\newline{}
\begin{itemize}
\item обогатил датасет
\item подобрал zero-shot classification модель
\item обучил модель-классификатор
\item обеспечил координацию работы команды
\item презентовал результаты
\end{itemize}}

\section{Выступления}
\cventry{25.06.2024}{Опыт обучения и применения нейросетей в качестве модуля российской DCAP системы}{Moscow Python Meetup}{Москва}{Компания Makves (входит в группу компаний «Гарда») разрабатывает российскую DCAP (data-centric audit and protection) систему для защиты корпоративных данных. Для анализа неструктурированных данных необходимо применять нейросети. В докладе я рассказал о проблемах, с которыми мы столкнулись при создании нейросетей, от этапа сбора и разметки данных и до создания нескольких микросервисов.}{}

\newpage

\section{Образование}
\cventry{2024}{Анализ данных на языке SQL}{УЦ «Специалист»}{Москва}{повышение квалификации}{}
\cventry{2022—2023}{Computer Vision Engineer}{Deep Learning School ФПМИ МФТИ}{Москва}{профессиональная переподготовка}{}
\cventry{2022}{Специалист по Data Science}{Яндекс Практикум}{Москва}{профессиональная переподготовка}{}
\cventry{2021—2022}{Введение в искусственный интеллект и нейросети для авиационных приложений}{МАИ}{Москва}{повышение квалификации}{}
\cventry{2005—2008}{Перевод и переводоведение}{МАИ}{Москва}{специалитет}{}
\cventry{2003—2009}{Авиационная и ракетно-космическая теплотехника}{МАИ}{Москва}{специалитет}{}

\section{Языки}
\cvskillentry{Русский}{5}{родной}{}{}
\cvskillentry{Английский}{4}{В2}{}{}
\cvskillentry{Немецкий}{4}{В2}{}{}
\cvskillentry{Эсперанто}{4}{В2}{}{}

\section{Навыки и технологии}
\begin{cvcolumns}
  \cvcolumn{}{
  	\begin{itemize}
  		\item Deep Learning
  		\item NLP, NER
  		\item Computer Vision
  		\item Speech Recognition
  		\item Machine learning
  		\item Anomaly Detection
  		\item Data analysis
  		\item Data visualisation
  		\item Statistics
  	\end{itemize}}
  \cvcolumn{}{
  	\begin{itemize}
  		\item Python
  		\item SQL
  		\item Linux
  		\item Docker
  		\item YOLO
  		\item Natasha
  		\item ONNX
  		\item HuggingFace
  		\item Ollama
  		\item U-Net
  		\item AirFlow
  		\item MLFlow
  		\item CVAT
  		\item Plotly
  	\end{itemize}}
  \cvcolumn{}{\begin{itemize}
  		\item Pytorch
  		\item Lightning
  		\item Pandas
  		\item NumPy
  		\item Sklearn
  		\item FastAPI
  		\item uvicorn
  		\item PyOD
  		\item PySAD
  		\item Optimum
  		\item pywin32
  		\item CatBoost
  		\item XGBoost
  		\item PostgreSQL
  	\end{itemize}}
\end{cvcolumns}
% \section{Сертификаты}
% \cvitem{Yandex}{Тренировки по ML}
% \cvitem{ФКН ВШЭ}{MLOps Bootcamp}
% \cvitem{Kaggle}{Feature Engineering}
% \cvitem{Kaggle}{Data Visualization}
\end{document}


%% end of file `template.tex'.
